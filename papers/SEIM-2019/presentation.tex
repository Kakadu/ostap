\documentclass[10pt, mathserif]{beamer}

\usepackage[utf8]{inputenc}
\usepackage[english,russian]{babel}
\usepackage[T2A]{fontenc}

\usepackage{listings}
\usepackage{color}
\usepackage{amssymb, amsmath}
\usepackage[all]{xy}
\usepackage{alltt}
\usepackage{pslatex}
\usepackage{epigraph}
\usepackage{verbatim}
\usepackage{graphicx}
\usepackage{latexsym}
\usepackage{array}
\usepackage{mathrsfs}

\renewcommand{\ttdefault}{cmtt}

\makeatletter
\newcolumntype{e}[1]{%--- Enumerated cells ---
   >{\minipage[t]{\linewidth}%
     \NoHyper%                Hyperref adds a vertical space
     \let\\\tabularnewline
     \enumerate
        \addtolength{\rightskip}{0pt plus 50pt}% for raggedright
        \setlength{\itemsep}{-\parsep}}%
   p{#1}%
   <{\@finalstrut\@arstrutbox\endenumerate
     \endNoHyper
     \endminipage}}

\newcolumntype{i}[1]{%--- Itemized cells ---
   >{\minipage[t]{\linewidth}%
        \let\\\tabularnewline
        \itemize
           \addtolength{\rightskip}{0pt plus 50pt}%
           \setlength{\itemsep}{-\parsep}}%
   p{#1}%
   <{\@finalstrut\@arstrutbox\enditemize\endminipage}}

\AtBeginDocument{%
    \@ifpackageloaded{hyperref}{}%
        {\let\NoHyper\relax\let\endNoHyper\relax}}
\makeatother

\definecolor{shadecolor}{gray}{1.00}
\definecolor{darkgray}{gray}{0.30}

\newcommand{\set}[1]{\{#1\}}
\newcommand{\angled}[1]{\langle {#1} \rangle}
\newcommand{\fib}{\rightarrow_{\mathit{fib}}}
\newcommand{\fibm}{\Rightarrow_{\mathit{fib}}}
\newcommand{\oo}[1]{{#1}^o}
\newcommand{\inml}[1]{\mbox{\lstinline{#1}}}

\setlength{\epigraphwidth}{.55\textwidth}

\definecolor{light-gray}{gray}{0.90}
\newcommand{\graybox}[1]{\colorbox{light-gray}{#1}}

\newcommand{\nredrule}[3]{
  \begin{array}{cl}
    \textsf{[{#1}]}&
    \begin{array}{c}
      #2 \\
      \hline
      \raisebox{-1pt}{\ensuremath{#3}}
    \end{array}
  \end{array}}

\newcommand{\naxiom}[2]{
  \begin{array}{cl}
    \textsf{[{#1}]} & \raisebox{-1pt}{\ensuremath{#2}}
  \end{array}}

\lstdefinelanguage{ocaml}{
keywords={ostap, let, begin, end, in, match, type, and, fun,
function, try, with, class, object, method, of, rec, repeat, until,
while, not, do, done, as, val, inherit, module, sig, @type, struct,
if, then, else, open, virtual, new, fresh},
sensitive=true,
basicstyle=\small\ttfamily,
commentstyle=\scriptsize\rmfamily,
keywordstyle=\underline,
identifierstyle=\ttfamily,
basewidth={0.5em,0.5em},
columns=fixed,
fontadjust=true,
literate={->}{{$\to$}}3
         {===}{{$\equiv$}}1
}

\lstdefinelanguage{scheme}{
keywords={define, conde, fresh},
sensitive=true,
basicstyle=\small,
commentstyle=\scriptsize\rmfamily,
keywordstyle=\ttfamily\bfseries,
identifierstyle=\ttfamily,
basewidth={0.5em,0.5em},
columns=fixed,
fontadjust=true,
literate={==}{{$\equiv$}}1
}

\lstset{
basicstyle=\small,
identifierstyle=\ttfamily,
keywordstyle=\bfseries,
commentstyle=\scriptsize\rmfamily,
basewidth={0.5em,0.5em},
fontadjust=true,
%escapechar=!,
language=ocaml,
mathescape=true
}

\setbeamertemplate{footline}[frame number]
\setbeamertemplate{navigation symbols}{}
\setbeamertemplate{blocks}[rounded][shadow=true]
\beamertemplateballitem
\usefont{T2A}{PTSans-TLF}{m}{n}

\mode<presentation>{
  \usetheme{default}
}

\theoremstyle{definition}

\title{Монадические парсер-комбинаторы с поддержкой симметричной альтерации и левой рекурсии}

\author{Данила Боровков}

\date{
   \vskip 3cm
   \small{
   \textbf{SEIM}\\
   13 апреля 2019 \\
   Санкт-Петербург}
}

\begin{document}
\begin{frame}
  \titlepage
\end{frame}

\begin{frame}[fragile]{Синтаксический анализ и парсер-комбинаторы}
  \begin{itemize}
    \item Синтаксический анализатор (или парсер) может быть реализован в виде монады
    \item Парсер-комбинаторы~--- подход к построению парсеров
      \begin{itemize}
         \item парсер как значение
         \item парсеры высших порядков
         \item поддержка специализированной пользовательсвой семантики
      \end{itemize}
   \item Классические трудности: левая рекурсия и несимметричная альтерация (``longest match first'')
  \end{itemize}
\end{frame}

\begin{frame}[fragile]{Левая рекурсия}

  Необходимость использования парсеров высших порядков для разработки библиотек переиспользуемых
  синтаксических компонент
  \vskip3mm

  Для парсеров высших порядков статическое определение левой рекурсии неразрешимо:
  \vskip3mm

  \begin{center}
    \texttt{A[B] : B ...}~--- леворекурсивен, если $\mathtt{A[B]} \equiv B$
  \end{center}
  \vskip3mm

  $\Rightarrow$ левая рекурсия не может быть устранена статически с помощью классических подходов

\end{frame}

\begin{frame}[fragile]{Несимметричная альтерация}

  В классических парсер-комбинаторах \texttt{A | B} работает, как ожидается, только если

  \[
  \forall\omega\in{\mathscr L}(\mathtt{A})\;\;\forall\psi\not=\epsilon\;\;\omega\psi\not\in{\mathscr L}(\mathtt{B})
  \]


  $\Rightarrow$ в выражении

  \begin{center}
    \verb/opnd (">=" | ">") opnd/
  \end{center}

  нельзя поменять местами \verb/">="/ и \verb/">"/

\end{frame}

\begin{frame}[fragile]{CPS + мемоизация}

  \textbf{Meerkat}~--- библиотека CF-полных парсер-комбинаторов на Scala (Izmaylova, Afroozeh, Van der Storm, PEPM-2015)
  \vskip3mm

  \begin{itemize}
    \item[+] нечувствительны к левой рекурсии и ``longest match first''
    \item[-] строят SPPF (нет пользовательской семантики)
    \item[-] не монадические
  \end{itemize}
  \vskip3mm

  $\Rightarrow$ требуется доработка

\end{frame}

\begin{frame}[fragile]{Библиотека \textbf{Ostap}}
  \begin{itemize}
    \item Direct-style монадические парсер-комбинаторы
    \item Синтаксическое расширение для \texttt{OCaml}
    \item Объектно-ориентированное представление входного потока
  \end{itemize}
  \vskip3mm
  \begin{lstlisting}
    ostap (
      list[elem] : hd:elem tl:elem* {hd :: tl};
      main       : list[ostap (IDENT)] -EOF
    )
  \end{lstlisting}

\end{frame}

\begin{frame}[fragile]{Переход от direct-style к CPS: типы}
  Было:
  \begin{lstlisting}
    type ($\alpha$, $\beta$) tag = Parsed of $\alpha$ * $\beta$ option | Failed of $\beta$ option
    type ($\sigma$, $\alpha$, $\beta$) result  = ($\alpha$ * $\sigma$, $\beta$) tag
    and  ($\sigma$, $\alpha$, $\beta$) parser  = $\sigma$ -> ($\sigma$, $\alpha$, $\beta$) result
  \end{lstlisting}

  Стало:
  \begin{lstlisting}
    type ($\alpha$, $\beta$) tag =
      Parsed of $\alpha$ * $\beta$ option
    | Failed of $\beta$ option
    | Empty
    type ($\sigma$, $\alpha$, $\beta$) result = ($\alpha$ * $\sigma$, $\beta$) tag
    type ($\alpha$, $\sigma$, $\beta$, $\gamma$) k = $\alpha$ -> $\sigma$ -> ($\sigma$, $\beta$, $\gamma$) result
    type ($\alpha$, $\sigma$, $\beta$, $\gamma$) parser = $\sigma$ -> ($\alpha$, $\sigma$, $\beta$, $\gamma$) k -> ($\sigma$, $\beta$, $\gamma$) result
  \end{lstlisting}

\end{frame}
%
% \begin{frame}[fragile]{CPS: комбинаторы}
%
%   \begin{lstlisting}
%    let alt x y =
%          (fun s k -> (x s k) < @ > (y s k))
%
%    let seq x y =
%      fun s k -> x s (         (fun a s' -> y a s' k))
%
%    let opt p =
%           (fun s k ->
%        let newk =          (fun a s -> k (Some a) s) in
%        (p s newk) < @ > (k None s))
%
%   \end{lstlisting}
%
% \end{frame}

\begin{frame}[fragile]{CPS: комбинаторы}

  \begin{lstlisting}
   let alt x y =
    $\only<2>{\boxed{\text{memo}}}$(fun s k -> (x s k) < @ > (y s k))

   let seq x y =
     fun s k -> x s $\only<2>{\text{(}\boxed{\text{memo\_k}}}$(fun a s' -> y a s' k)$\only<2>{\text{)}}$

   let opt p =
    $\only<2>{\boxed{\text{memo}}}$(fun s k ->
       let newk =$\only<2>{\boxed{\text{memo\_k}}}$(fun a s -> k (Some a) s) in
       (p s newk) < @ > (k None s))

  \end{lstlisting}

\end{frame}
%
% \begin{frame}[fragile]{CPS: мемоизация}
%
%   \begin{lstlisting}[basicstyle=\small]
%     let memo_result = fun p ->
%       let ss : ('stream * 'a) list ref = ref      [] in
%       let ks :                K.ks ref = ref K.empty in
%       fun k ->
%         if $\mbox{это первый запуск p}$
%         then (
%           ks := K.singleton k;
%           p (fun a s ->
%               $\mbox{если (s, a) не было в ss}$
%               $\mbox{записываем ее в ss и}$
%               $\mbox{запускаем на ней все продолжения из ks}$
%               $\mbox{иначе возвращаем Empty}$
%             )
%         )
%         else (
%           ks := K.add k !ks;
%           $\mbox{запустить k на всех сохраненных в ss парах}$
%         )
%   \end{lstlisting}
%
% \end{frame}

% \begin{frame}[fragile]{CPS: мемоизация}
%
% \begin{lstlisting}[basicstyle=\small]
%   p = p 'a' | 'a'
%
%   memo:
%
%     ss:
%     ks:
%
%   p "aaab" k_0
% \end{lstlisting}
%
% \end{frame}
%
% \begin{frame}[fragile]{CPS: мемоизация}
%
% \begin{lstlisting}[basicstyle=\small]
%   p = p 'a' | 'a'
%
%   memo: (p' "aaab", "aaab")
%
%     ss:
%     ks: k_0
%
%   p "aaab" k_ss
% \end{lstlisting}
%
% \end{frame}

% \begin{frame}[fragile]{CPS: мемоизация}
%
% \begin{lstlisting}[basicstyle=\small]
%   p = p 'a' | 'a'
%
%   memo: (p' "aaab", "aaab")
%
%     ss:
%     ks: k_0
%
%   'a' "aaab" k_ss
%
%   (p 'a') "aaab" k_ss
% \end{lstlisting}
%
% \end{frame}
%
% \begin{frame}[fragile]{CPS: мемоизация}
%
% \begin{lstlisting}[basicstyle=\small]
%   p = p 'a' | 'a'
%
%   memo: (p' "aaab", "aaab")
%
%     ss:
%     ks: k_0
%
%   'a' "aaab" k_ss $\Rightarrow$ k_ss "a" "aab"
%
%   (p 'a') "aaab" k_ss
% \end{lstlisting}
%
% \end{frame}

% \begin{frame}[fragile]{CPS: мемоизация}
%
% \begin{lstlisting}[basicstyle=\small]
%   p = p 'a' | 'a'
%
%   memo: (p' "aaab", "aaab")
%
%     ss: ("a", "aab")
%     ks: k_0
%
%   'a' "aaab" k_ss $\Rightarrow$ k_ss "a" "aab" $\Rightarrow$ k_0 "a" "aab"
%
%   (p 'a') "aaab" k_ss
% \end{lstlisting}
%
% \end{frame}
%
% \begin{frame}[fragile]{CPS: мемоизация}
%
% \begin{lstlisting}[basicstyle=\small]
%   p = p 'a' | 'a'
%
%   memo: (p' "aaab", "aaab")
%
%     ss: ("a", "aab")
%     ks: k_0
%
%   'a' "aaab" k_ss $\Rightarrow$ k_ss "a" "aab" $\Rightarrow$ k_0 "a" "aab" $\Rightarrow$ error
%
%   (p 'a') "aaab" k_ss
% \end{lstlisting}
%
% \end{frame}
%
% \begin{frame}[fragile]{CPS: мемоизация}
%
% \begin{lstlisting}[basicstyle=\small]
%   p = p 'a' | 'a'
%
%   memo: (p' "aaab", "aaab")
%
%     ss: ("a", "aab")
%     ks: k_0
%
%   'a' "aaab" k_ss $\Rightarrow$ k_ss "a" "aab" $\Rightarrow$ k_0 "a" "aab" $\Rightarrow$ error
%
%   (p 'a') "aaab" k_ss $\Rightarrow$ p "aaab" (fun a s -> 'a' s k_ss)
% \end{lstlisting}
%
% \end{frame}
%
% \begin{frame}[fragile]{CPS: мемоизация}
%
% \begin{lstlisting}[basicstyle=\small]
%   p = p 'a' | 'a'
%
%   memo: (p' "aaab", "aaab")
%
%     ss: ("a", "aab")
%     ks: k_0, (fun s -> 'a' s k_ss) = k_1
%
%   'a' "aaab" k_ss $\Rightarrow$ k_ss "a" "aab" $\Rightarrow$ k_0 "a" "aab" $\Rightarrow$ error
%
%   (p 'a') "aaab" k_ss $\Rightarrow$ p "aaab" (fun a s -> 'a' s k_ss) $\Rightarrow$
%   $\Rightarrow$ (fun a s -> 'a' s k_ss) "a" "aab"
% \end{lstlisting}
%
% \end{frame}
%
% \begin{frame}[fragile]{CPS: мемоизация}
%
% \begin{lstlisting}[basicstyle=\small]
%   p = p 'a' | 'a'
%
%   memo: (p' "aaab", "aaab")
%
%     ss: ("a", "aab")
%     ks: k_0, (fun s -> 'a' s k_ss) = k_1
%
%   'a' "aaab" k_ss $\Rightarrow$ k_ss "a" "aab" $\Rightarrow$ k_0 "a" "aab" $\Rightarrow$ error
%
%   (p 'a') "aaab" k_ss $\Rightarrow$ p "aaab" (fun a s -> 'a' s k_ss) $\Rightarrow$
%   $\Rightarrow$ (fun a s -> 'a' s k_ss) "a" "aab" $\Rightarrow$
%   k_ss "aa" "ab"
% \end{lstlisting}
%
% \end{frame}
%
% \begin{frame}[fragile]{CPS: мемоизация}
%
% \begin{lstlisting}[basicstyle=\small]
%   p = p 'a' | 'a'
%
%   memo: (p' "aaab", "aaab")
%
%     ss: ("a", "aab"), ("aa", "ab")
%     ks: k_0, (fun s -> 'a' s k_ss) = k_1
%
%   'a' "aaab" k_ss $\Rightarrow$ k_ss "a" "aab" $\Rightarrow$ k_0 "a" "aab" $\Rightarrow$ error
%
%   (p 'a') "aaab" k_ss $\Rightarrow$ p "aaab" (fun a s -> 'a' s k_ss) $\Rightarrow$
%   $\Rightarrow$ (fun a s -> 'a' s k_ss) "a" "aab" $\Rightarrow$
%   k_ss "aa" "ab" $\Rightarrow$ k_0 "aa" "ab"
%         $\Downarrow$
%   k_1 "aa" "ab"
% \end{lstlisting}
%
% \end{frame}

% \begin{frame}[fragile]{CPS: мемоизация}
%
% \begin{lstlisting}[basicstyle=\small]
%   p = p 'a' | 'a'
%
%   memo: (p' "aaab", "aaab")
%
%     ss: ("a", "aab"), ("aa", "ab")
%     ks: k_0, (fun s -> 'a' s k_ss) = k_1
%
%   'a' "aaab" k_ss $\Rightarrow$ k_ss "a" "aab" $\Rightarrow$ k_0 "a" "aab" $\Rightarrow$ error
%
%   (p 'a') "aaab" k_ss $\Rightarrow$ p "aaab" (fun a s -> 'a' s k_ss) $\Rightarrow$
%   $\Rightarrow$ (fun a s -> 'a' s k_ss) "a" "aab" $\Rightarrow$
%   k_ss "aa" "ab" $\Rightarrow$ k_0 "aa" "ab" $\Rightarrow$ error
%         $\Downarrow$
%   k_1 "aa" "ab" $\Rightarrow$ k_ss "aaa" "b"
% \end{lstlisting}
%
% \end{frame}
%
% \begin{frame}[fragile]{CPS: мемоизация}
%
% \begin{lstlisting}[basicstyle=\small]
%   p = p 'a' | 'a'
%
%   memo: (p' "aaab", "aaab")
%
%     ss: ("a", "aab"), ("aa", "ab")
%     ks: k_0, (fun s -> 'a' s k_ss) = k_1
%
%   'a' "aaab" k_ss $\Rightarrow$ k_ss "a" "aab" $\Rightarrow$ k_0 "a" "aab" $\Rightarrow$ error
%
%   (p 'a') "aaab" k_ss $\Rightarrow$ p "aaab" (fun a s -> 'a' s k_ss) $\Rightarrow$
%   $\Rightarrow$ (fun a s -> 'a' s k_ss) "a" "aab" $\Rightarrow$
%   k_ss "aa" "ab" $\Rightarrow$ k_0 "aa" "ab" $\Rightarrow$ error
%         $\Downarrow$
%   k_1 "aa" "ab" $\Rightarrow$ k_ss "aaa" "b" $\Rightarrow$ k_0 "aaa" "b"
%                          $\Downarrow$
%                     k_1 "aaa" "b"
% \end{lstlisting}
%
% \end{frame}

\begin{frame}[fragile]{CPS: мемоизация}

\begin{lstlisting}[basicstyle=\small]
  p = p 'a' | 'a'

  memo: $\only<2->{\texttt{(p' ``aaab'', ``aaab'')}}$

    ss: $\only<5->{\texttt{(a, ``aab'')}} \only<13->{\texttt{, (aa, ``ab'')}} \only<17->{\texttt{, (aaa, ``b'')}}$
    ks: $\only<2->{\texttt{k\_0}} \only<9->{\texttt{, (fun a s -> 'a' s k\_ss) = k\_1}}$

  $\only<3->{\texttt{'a' ``aaab'' k\_ss}} \only<4->{\Rightarrow \texttt{k\_ss a ``aab''}} \only<6->{\Rightarrow \texttt{k\_0 a ``aab''}} \only<7->{\Rightarrow \texttt{Failure}}$
  $\only<1>{\texttt{p ``aaab'' k\_0}} \only<2>{\texttt{p ``aaab'' k\_ss}}$
  $\only<3->{\texttt{(p 'a') ``aaab'' k\_ss}} \only<8->{\Rightarrow \texttt{p ``aaab'' (fun a s -> 'a' s k\_ss)}} \only<9->{\Rightarrow}$
  $\only<10->{\Rightarrow \texttt{(fun a s -> 'a' s k\_ss) a ``aab''}} \only<11->{\Rightarrow}$
  $\only<12->{\texttt{k\_ss aa ``ab''}} \only<14->{\Rightarrow \texttt{k\_0 aa ``ab''}} \only<15->{\Rightarrow \texttt{Failure}}$
        $\only<14->{\Downarrow}$
  $\only<14->{\texttt{k\_1 aa ``ab''}} \only<16->{\Rightarrow \texttt{k\_ss aaa ``b''}} \only<18->{\Rightarrow \texttt{k\_0 aaa ``b''}} \only<19->{\Rightarrow \texttt{Failure}}$
                           $\only<18->{\Downarrow}$
                    $\only<18->{\texttt{k\_1 aaa ``b''}} \only<20->{\Rightarrow \texttt{Failure}}$
\end{lstlisting}

\end{frame}

\begin{frame}[fragile]{Остальное}

  \begin{itemize}
     \item Явно-полиморфные типы методов входного потока:

       \begin{lstlisting}
  <
    getCONST : $\beta$ . ($\alpha$ -> 'self -> ('self, $\beta$, $\gamma$) result) ->
                         ('self, $\beta$, $\gamma$) result;
    ..
  >
       \end{lstlisting}

     \item Генерируемая неподвижная точка для рекурсивных определений:

       \begin{lstlisting}
         ostap (x : "a"; y[z] : y[z] "+" x | x)

                           $\Downarrow$

         let (x, y) =
           let generated_fixpoint = ... in
           let x'   = fun x y -> ostap ("a")
           and y' z = fun x y -> ostap (y[z] "+" x | x)
         in generated_fixpoint x' y'
       \end{lstlisting}

  \end{itemize}

\end{frame}

\begin{frame}[fragile]{Апробация}
  \begin{itemize}
     \item Проведено регрессионное тестирование на оригинальных тестах Ostap, которые покрывают всю функциональность библиотеки
     \vskip3mm

     \item Апробировано на учебном компиляторе
  \end{itemize}
\end{frame}

\begin{frame}[fragile]{Апробация: несимметричность альтерации}

  Если переставить \lstinline|">="| и \lstinline|">"|, то парсер останется работоспособным и время работы не изменится.

  \begin{lstlisting}[basicstyle=\small]
    Ostap.Util.expr [|
      `Lefta, [$\verb/"!!"/$];
      `Lefta, [$\verb/"&&"/$];
      `Nona , [$\verb/"=="/$; $\verb/"!="/$; $\verb/"<="/$; $\verb/"<"/$; $\verb/">="/$; $\verb/">"/$];
      `Lefta, [$\verb/"++"/$; $\verb/"+"/$; $\verb/"-"/$];
      `Lefta, [$\verb/"*"/$; $\verb/"//"$; $\verb/"%"/$]
    |]
  \end{lstlisting}
  \vskip3mm
\end{frame}

\begin{frame}[fragile]{Апробация: левая рекурсия}

  Время работы не изменилось, но появилась возможность использовать леворекурсивные описания.
  \vskip3mm
  \begin{lstlisting}[basicstyle=\small]
    parse:
          s:stmt ";" ss:parse {Seq (s, ss)}
        | stmt;

                   $\Downarrow$

    parse:
          ss:parse ";" s:stmt {Seq (ss, s)}
        | stmt;

  \end{lstlisting}

\end{frame}

\begin{frame}[fragile]{Апробация: производительность}

Тестовый набор вида \lstinline|"x1 := {expr}; ... xn := {expr}"|
\vskip3mm

  \begin{table}[htbp]
  \begin{center}
  \begin{tabular}{|c|c|c|}
  \hline
  \textbf{Размер} & \multicolumn{2}{|c|}{\textbf{Время разбора, c}} \\
  \cline{2-3}
  \textbf{примера} & \textbf{Старая версия} & \textbf{Новая версия} \\
  \hline
  100& 0,25& 0,13 \\
  \hline
  200& 0,52& 0,24 \\
  \hline
  400& 2,13& 0,56 \\
  \hline
  800& 9,19& 1,97 \\
  \hline
  1200& 21,93& 4,61 \\
  \hline
  1600& 42,65& 9,50 \\
  \hline
  \end{tabular}
  \vskip3mm
  %\caption{Результаты сравнения производительности старой и новой реализаций библиотеки}
  \label{tab1}
  \end{center}\vspace{-10mm}
  \end{table}

\end{frame}

\end{document}
