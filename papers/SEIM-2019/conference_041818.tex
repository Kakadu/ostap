\documentclass[conference]{IEEEtran}
\IEEEoverridecommandlockouts
% The preceding line is only needed to identify funding in the first footnote. If that is unneeded, please comment it out.
\usepackage{cite}
\usepackage{amsmath,amssymb,amsfonts}
\usepackage{algorithmic}
\usepackage{graphicx}
\usepackage{textcomp}
\usepackage{listings}
\usepackage{verbatim}

\lstdefinelanguage{ocaml}{
keywords={ostap, let, begin, end, in, match, type, and, fun,
function, try, with, class, object, method, of, rec, repeat, until,
when, while, not, do, done, as, val, inherit, module, sig, @type, struct,
if, then, else, open, virtual, new, fresh},
sensitive=true,
basicstyle=\normalsize\ttfamily,
commentstyle=\scriptsize\rmfamily,
keywordstyle=\underline,
identifierstyle=\ttfamily,
basewidth={0.5em,0.5em},
columns=fixed,
fontadjust=true,
literate={->}{{$\to$}}3
         {===}{{$\equiv$}}1
}

\lstset{
extendedchars=\true,
numbers=left,
numbersep=-4pt,
basicstyle=\normalsize,
identifierstyle=\ttfamily,
keywordstyle=\bfseries,
commentstyle=\scriptsize\rmfamily,
basewidth={0.5em,0.5em},
fontadjust=true,
%escapechar=!,
language=ocaml,
mathescape=true
}

\setlength{\abovedisplayskip}{1pt}
\setlength{\belowdisplayskip}{1pt}

\usepackage{xcolor}
\def\BibTeX{{\rm B\kern-.05em{\sc i\kern-.025em b}\kern-.08em
    T\kern-.1667em\lower.7ex\hbox{E}\kern-.125emX}}


\usepackage[utf8]{inputenc}
\usepackage[russian]{babel}

\pagestyle{empty}

\begin{document}

\title{Монадические парсер-комбинаторы с поддержкой симметричной альтерации и левой рекурсии}

\maketitle

\begin{abstract}
  В данной работе описывается модификация монадических парсер-комбинаторов, которая обеспечивает симметричность
  альтерации и поддержку левой рекурсии. Обеспечение этих свойств особенно важно в контексте компонентной разработки
  синтаксических анализаторов. Описаны основные понятия, подходы и проблемы в данной области,
  представлена реализация библиотеки монадических парсер-комбинаторов, поддерживающих левую рекурсию и симметричную альтерацию,
  и приведены результаты как количественной, так и качественной их апробации.
  %С технической точки зрения данная
  %библиотека представляет собой альтернативную реализацию уже существующего инструмента синтаксического анализа,
  %совместимую с ним по формату входных спецификаций.
\end{abstract}

\begin{IEEEkeywords}
Objective Caml, монадические парсер-комбинаторы, левая рекурсия, мемоизация, стиль передачи продолжений
\end{IEEEkeywords}

\section{Введение}

Монадические парсер-комбинаторы~\cite{meijer,wadler} представляют собой удобный механизм синтаксического анализа, который, по сравнению
с обычными генеративными подходами, позволяет бесшовно интегрировать синтаксический анализатор в основной код используемого языкового инструмента, а
также применять компонентизацию при разработке синтаксических анализаторов. Монадичность естественным образом поддерживает пользовательскую
семантику, что позволяет сразу строить необходимое внутреннее представление, минуя стадии промежуточного построения дерева разбора.

Однако каноническая реализация монадических парсер-комбинаторов~\cite{meijer} обладает двумя фундаментальными недостатками: чувствительностью
к левой рекурсии и несимметричностью альтерации. При этом до недавнего времени~\cite{meerkat} все способы устранения этих недостатков были
несовместимы с требованием компонентной разработки.

Преобразование устранения левой рекурсии, хорошо известное из теории формальных языков, не может считаться приемлемым решением, так как,
формально сохраняя распознаваемый язык прежним, оно, фактически, искажает порядок исполнения примитивов пользовательской семантики, заменяя
построение левоассоциативных структур данных правоассоциативными, что требует отдельного прохода для перебалансировки. Кроме того,
левая рекурсия, отсутствующая на уровне индивидуальных компонент, может появиться в результате их безобидного на первый взгляд сочетания.
Предположим, что у нас есть компонента, в которой определен парсер высшего порядка, распознающий последовательность операторов (каждый из которых
распознается парсером-параметром \lstinline|``s''|):

\begin{lstlisting}[basicstyle=\small]
  seq[s]: s (";" s)*
\end{lstlisting}

Этот парсер высшего порядка может быть использован для построения парсера операторов какого-либо конкретного языка:

\begin{lstlisting}[basicstyle=\small]
  stmt: cond[stmt] | loop[stmt] | seq[stmt]
\end{lstlisting}

Как можно увидеть, парсер \lstinline|``stmt''| оказался леворекурсивным. Таким образом, требование компонентизации не позволяет
использовать статические методы для устранения левой рекурсии.

Еще одной проблемой канонических парсер-комбинаторов является несимметричность альтерации: выражения $A \mid B$ и $B\mid A$
неэквивалентны, и, как правило, только одна из этих форм правильно специфицирует язык. Условие, которому должна удовлетворять
``правильная'' альтерация, таково:

\[
\forall\phi\;\psi\;:\; A\rightarrow\phi \wedge B\rightarrow\psi \Rightarrow \forall\omega\ne\epsilon\;:\;\psi\ne\phi\omega
\]

Иными словами, первый член альтерации не должен распознавать ни один собственный префикс ни одного слова, распознаваемого
вторым членом (неформально это правило можно выразить девизом ``longest match first''). В следующем разделе мы объясним происхождение такого феномена,
а пока заметим, что в случае замкнутого описания можно подобрать правильный порядок альтернатив. Однако при переходе к компонентному случаю
ситуация меняется: даже если парсеры $A$ и $B$, реализованные в разных компонентах, оба удовлетворяют инварианту ``longest match first'', может оказаться,
что как $A\mid B$, так и $B\mid A$ его нарушают, что делает невозможным сборку правильного парсера без изменения реализации компонент.

В данной работе представлена реализация монадических парсер-комбинаторов, которые нечувствительны к левой рекурсии и порядку альтернатив. Был переиспользован
существующий подход на основе передачи продолжений и мемоизации~\cite{meerkat} и расширен до монадического случая.

\section{Контекст работы}

В данном разделе будут описаны основные понятия и подходы, которые сложились в области комбинаторного синтаксического анализа, а также сформулируем ряд важных проблем и
способов их решения.

\subsection{Монадические парсер-комбинаторы}

Парсер-комбинаторы в широком смысле можно отнести к ряду наиболее устойчивых идиом в функциональном программировании. Техника парсер-комбинаторов опирается на идею
представления нетерминалов функциями, которые их распознают, а грамматических примитивов~--- функциями высшего порядка (комбинаторами), которые конструируют новые
распознаватели из существующих. В более узком смысле понятие \emph{монадических} парсер-комбинаторов было введено в работах~\cite{meijer,wadler}. Мы рассмотрим здесь ряд вариантов
этой техники, начиная с простейших распознавателей и кончая собственно монадическими парсер-комбинаторами, поддерживающими пользовательскую семантику. Далее для простоты
мы будем считать, что входной поток для анализа представлен списком символов.

В простейшем случае распознаватель~--- это функция, которая получает входной поток и пытается нечто распознать в его начале. Важным соображением является то, что, во-первых,
входной поток может не содержать того, что ожидалось, и, во-вторых, распознавание, вообще говоря, может быть неоднозначным. Таким образом появляется естественный ``контракт''
на тип распознавателя~--- это функция, которая преобразует входной поток в список остаточных потоков. Если этот список пуст, то распознавание окончилось неудачей.
Рассмотрим некоторые элементарные распознаватели:

\begin{lstlisting}[basicstyle=\small]
  let empty s = [s]
  let sym c = function
  | s :: ss when s = c -> [ss]
  | _ -> []
\end{lstlisting}

Здесь \lstinline|empty| соответствует распознавателю пустой строки, а \lstinline|sym с|~--- распознавателю единственного символа \lstinline|``c''|. Простейшими комбинаторами
являются комбинаторы последовательного разбора \lstinline|seq| и альтернативы \lstinline|alt|:

\begin{lstlisting}[basicstyle=\small]
  let seq x y s = flatten @@ map y (x s)
  let alt x y s = x s @ y s
\end{lstlisting}

Здесь \lstinline|flatten|~--- функция, преобразующая список списков в простой список, \lstinline|@@|~--- инфиксное применение, \lstinline|@|~--- конкатенация. С помощью
этих двух основных распознавателей и двух комбинаторов потенциально можно описать распознаватель достаточно широкого класса языков. С практической точки зрения, однако,
представляют интерес не распознаватели, а анализаторы, которые не просто проверяют соответствие входного потока грамматике, но и вычисляют некоторое значение в результате
такого сопоставления. Не составляет труда, используя данную идею, переписать два основных распознавателя:

\begin{lstlisting}[basicstyle=\small]
  let empty v s = [(v, s)]
  let sym c = function
  | s :: ss when s = c -> [(c, ss)]
  | _ -> []
\end{lstlisting}

Теперь наравне с остатком входного потока каждый примитив возвращает семантическое значение~--- некоторое пустое значение \lstinline|``v''| в случае \lstinline|empty| и собственно распознанный
символ в случае \lstinline|sym|. Кроме того, следующий новый комбинатор позволяет использовать пользовательскую семантику, заданную функцией:

\begin{lstlisting}[basicstyle=\small]
  let sema x f s =
    map (fun (a, ss) -> (f a, ss)) (x s)
\end{lstlisting}

Здесь \lstinline|``x''|~--- парсер, \lstinline|``f''|~--- заданная пользователем в виде функции семантика, \lstinline|``s''|~--- входной поток. Фактически, этот комбинатор
применяет пользовательскую функцию к каждому семантическому значению, которое возвращает парсер \lstinline|``x''|.

Интересно, что реализация комбинатора \lstinline|alt| сохраняется в прежнем виде. Что же касается комбинатора \lstinline|seq|, то для его реализации существует несколько
вариантов. Один из них, собственно, и называется монадическим:

\begin{lstlisting}[basicstyle=\small]
  let seq x y s =
    flatten @@ map (fun (a, ss) -> y a ss) (x s)
\end{lstlisting}

Здесь \lstinline|``y''|~--- парсер высшего порядка, который параметризован семантическим значением, которое вычисляет парсер \lstinline|``x''|. С точки зрения монадической
идиомы \lstinline|seq| выполняет роль \lstinline|bind|, \lstinline|empty|~--- \lstinline|return|, а \lstinline|alt|~--- \lstinline|plus|.

Описанные здесь простейшие монадические комбинаторы обладают рядом известных недостатков. Во-первых, они не допускают использования левой рекурсии. Действительно, описание
вида \lstinline|let rec x s = (seq x ...) s| означает, что анализ входного потока с помощью парсера \lstinline|``x''| сводится к анализу этого же потока с помощью этого
же парсера. Иными словами, леворекурсивные описания в стиле монадических парсер-комбинаторов зацикливаются. Кроме того, сложность разбора в худшем случае является экспоненциальной
из-за реализации комбинатора \lstinline|alt|. Этот недостаток может быть исправлен альтернативной реализацией, при которой предпочтение отдается первому парсеру:

\begin{lstlisting}[basicstyle=\small]
  let alt x y s =
    match x s with
    | [] -> y s
    | z  -> z
\end{lstlisting}

Но в этом случае описание парсеров становится чувствительным к порядку альтернатив: действительно, если нечто, распознаваемое \lstinline|``x''|, является собственным префиксом
чего-то, распознаваемого \lstinline|``y''|, то до применения \lstinline|``y''| дело не дойдет и анализ, скорее всего, закончится неочевидной и непонятной для
пользователя ошибкой. Поэтому на практике для применения монадических парсер-комбинаторов сложилось два правила~--- не использовать левую рекурсию и придерживаться
принципа ``longest match first'': в альтернативе первым всегда должен идти парсер, который распознает более длинную подстроку входного потока. Заметим, что данные недостатки
характерны и для парсер-комбинаторов в целом, безотносительно их монадичности.

\subsection{Мемоизация и передача продолжений}

Подходам к решению сформулированных в конце предыдущего раздела проблем комбинаторного разбора посвящена обширная литература. Так, существуют парсер-комбинаторы,
которые используют алгоритмы, способные распознавать все контекстно-свободные грамматики (например, алгоритм Эрли~\cite{earleycomb} или обобщенный левосторонний разбор
(GLL)~\cite{GLLcomb}). Эти комбинаторы нечувствительны ни к левой рекурсии, ни к порядку следования альтернатив. Однако интеграция пользовательских семантик и монадичности
в данные комбинаторы является открытой проблемой, равно как и обеспечение композиционности описаний. С другой стороны, предпринимались попытки применить устранение
левой рекурсии к комбинаторному описанию~\cite{leftcorner}. Однако в ситуации парсеров высшего порядка само установление факта левой рекурсии является алгоритмически
неразрешимой проблемой: парсер \lstinline|let p x = seq x ...| леворекурсивен, если \lstinline|p x = x|. В работе~\cite{frost} поддержка левой рекурсии достигалась тем,
что цепь леворекурсивных вызовов искусственно прерывалась в тот момент, когда её глубина превосходила длину входного потока. Однако не всегда длина входного
потока известна статически.

Другим естественным направлением является использование мемоизации (собственно, мемоизацию использует и алгоритм Эрли). В ряде работ~\cite{tratt,warth,ford2002packrat,ford2004parsing}
исследовался формализм PEG (Parsing Expression Grammars), для которого возможна комбинаторная реализация, и было показано, что с его помощью можно распознавать
в том числе и леворекурсивные определения. Однако в основе данного подхода лежит принципиальное использование несимметричной альтерации, что серьезно затрудняет
компонентизацию.

Еще одним подходом является одновременное использование мемоизации и продолжений. В работе~\cite{swierstra2001combinator} предлагаются комбинаторы, исправляющие
ошибки. Однако данные комбинаторы не поддерживают левую рекурсию. Наконец, в работе~\cite{memo} показано, как с помощью мемоизации и передачи продолжений построить
алгоритм анализа, распознающий все КС-языки. На основе этого алгоритма сравнительно недавно была разработана библиотека парсер-комбинаторов Meerkat~\cite{meerkat} для языка Scala.
Анализаторы, построенные с помощью этой библиотеки, возвращают сжатый лес разбора (SPPF, Shared Packed Parse Forest). Полученные комбинаторы нечувствительны к порядку
альтернатив и левой рекурсии, однако не поддерживают пользовательскую семантику и не являются монадическими.

\subsection{Библиотека Ostap}

Библиотека Ostap~\cite{ostap} представляет собой реализацию монадических парсер-комбинаторов для языка OCaml. При её разработке преследовалась цель обеспечить возможность
сборки парсеров из раздельно компилируемых строго типизированных компонент. Для избежания неудобств, связанных с использованием парсер-комбинаторов напрямую, библиотека определяет
синтаксическое расширение языка OCaml, предоставляющее предметно-ориентированный язык (DSL, Domain Specific Language) для описания парсеров. Это расширение очень схоже с формой Бэкуса-Наура для описания грамматик, оно интуитивно понятно
и очень просто в использовании. Далее представлен пример парсера, написанный с помощью DSL библиотеки Ostap:

\begin{lstlisting}[basicstyle=\small]
   ostap (
     primary: c:IDENT {`Var c};
     exp:
        e:primary "+" p:exp {`Add (e, p)}
      | primary;
     main: exp -EOF
   )
\end{lstlisting}

Здесь определен парсер выражений, состоящих из идентификаторов и правоассоциативной операции сложения. При описании парсеров пользователь может задать свою семантику разбора в фигурных скобках.
Для целей компонентизации в библиотеке Ostap используется объектно-ориентированное представление входного потока, и для описания разбора токенов \lstinline|IDENT| и \lstinline|EOF| в данном
примере у объекта входного потока, переданного этому парсеру, должны быть реализованы методы \lstinline|getIDENT| и \lstinline|getEOF|, которые описывают лексический анализ для соответствующих
лексем. Учитывая тот факт, что типизация объектов в OCaml структурна (а не номинальна), такой подход дает возможность описывать парсеры-компоненты, не имеющие общего типа входного потока.
При комбинировании этих компонент тип требуемого входного потока выводится автоматически.

Библиотека Ostap была использована для реализации компилятора семейства обероноподобных языков~\cite{boulytchev2015combinators}; при этом, действительно, готовые анализаторы собирались из
прекомпилированных компонент.

\section{Реализация новых комбинаторов}

Предложенные в данной работе парсер-комбинаторы совмещают в себе, с одной стороны, использование продолжений и мемоизацию со стороны Meerkat, и, с другой~--- монадичность и пользовательскую семантику со стороны Ostap.
Это привело к тому, что новые комбинаторы обладают рядом отличительных черт по сравнению с обеими исходными библиотеками:

\begin{itemize}
\item по сравнению с Meerkat:

\begin{itemize}
\item изменился тип продолжений из-за необходимости обеспечения передачи монадических значений;
\item мемоизация учитывает монадические значения и параметры парсеров высшего порядка.
\end{itemize}

\item по сравнению с Ostap:

\begin{itemize}
\item все базовые комбинаторы переписаны в форму передачи продолжений;
\item введены явно-полиморфные типы методов для объектов, представляющих входной поток;
\item использована генерируемая неподвижная точка для группы взаимно-рекурсивных определений парсеров.
\end{itemize}
\end{itemize}

Далее мы более подробно опишем реализацию существенных составных частей новых комбинаторов.

\subsection{Типы}

При определении типов мы отталкивались от реализации парсер-комбинатора \lstinline|seq|, потому что именно в нем выражается монадичность парсеров и наиболее явно используются продолжения
(реализация комбинатора \lstinline|alt| интереса не представляет, потому что она не была изменена при адаптации алгоритма). Реализация \lstinline|seq| в библиотеке Meerkat
предполагает передачу второго парсера в качества аргумента первому:

\begin{lstlisting}[basicstyle=\small]
   fun x y s k -> x s (kmemo (fun s' -> y s' k))
\end{lstlisting}

Здесь \lstinline|``x''| и \lstinline|``y''|~--- парсеры, выступающие в качестве аргументов \lstinline|seq|, \lstinline|``s''|~--- входной поток, \lstinline|``k''|~--- продолжение,
\lstinline|kmemo|~--- примитив мемоизации.

В монадическом случае первый парсер должен передавать второму свой результат с помощью монадического значения. Так как первый парсер передает управление второму с
помощью вызова продолжения, то при этом вызове необходимо передавать и монадическое значение. Поэтому для продолжений был добавлен еще один аргумент~--- монадическое
значение \lstinline|``a''|, и комбинатор \lstinline|seq| приобрел следующий вид:

\begin{lstlisting}[basicstyle=\small]
   fun x y s k -> x s (kmemo (fun a s' -> y a s' k))
\end{lstlisting}

Теперь перейдем к описанию типов. Прежде всего, введем следующий тип:

\begin{lstlisting}[basicstyle=\small]
   type ($\alpha$, $\beta$) tag =
     Parsed of $\alpha$ * $\beta$ option
   | Failed of $\beta$ option
   | Empty
\end{lstlisting}

Этот тип описывает три возможных результата применения парсера к входному потоку. \lstinline|Parsed| означает успешный результат анализа и содержит некоторую информацию, полученную в процессе
анализа, и возможную информацию об ошибке. \lstinline|Failed| означает неудачу и содержит возможную информацию об ошибке. \lstinline|Empty| означает, что ничего не произошло, парсер не был запущен.
Далее этот тип будет использоваться специальным образом, для чего введем следующий синоним:

\begin{lstlisting}[basicstyle=\small]
   type ($\sigma$, $\alpha$, $\beta$) result = ($\alpha$ * $\sigma$, $\beta$) tag
\end{lstlisting}

Здесь $\alpha$~--- это тип значения, полученного в процессе анализа с помощью пользовательской семантики, $\sigma$~--- тип входного потока, $\beta$~--- тип возможной ошибки.

Тип продолжения \lstinline|k| вводится следующим образом:

\begin{lstlisting}[basicstyle=\small]
   type ($\alpha$, $\sigma$, $\beta$, $\gamma$) k =
      $\alpha$ -> $\sigma$ -> ($\sigma$, $\beta$, $\gamma$) result
\end{lstlisting}

Продолжение~--- это функция, которая принимает два аргумента и возвращает результат, тип которого был описан ранее. Первый из этих аргументов, $\alpha$~--- это тип монадического значения,
которое передается от текущего парсера следующему. Второй аргумент, $\sigma$~--- это остаток входного потока, который не был разобран предыдущим парсером и на котором должен быть
запущен следующий парсер.

Наконец, тип парсера определяется так:

\begin{lstlisting}[basicstyle=\small]
   type ($\alpha$, $\sigma$, $\beta$, $\gamma$) parser =
      $\sigma$ -> ($\alpha$, $\sigma$, $\beta$, $\gamma$) k -> ($\sigma$, $\beta$, $\gamma$) result
\end{lstlisting}

Парсер~--- это функция от двух аргументов: входного потока и продолжения. Здесь $\sigma$~--- тип входного потока, $\alpha$~--- тип монадического значения, $\beta$~--- тип результата, $\gamma$~---
тип ошибки.

\subsection{Мемоизация монадических парсер-комбинаторов}

Мемоизация в библиотеке Meerkat использует три функции: \lstinline|kmemo|, \lstinline|memo| и \lstinline|memoresult|.

Функция \lstinline|kmemo| имеет один аргумент~--- продолжение~--- и обеспечивает, что это продолжение будет вызвано на одном и том же наборе аргументов только один раз. Для этого используется
таблица для хранения результатов предыдущих запусков продолжения.

Функция \lstinline|memo| имеет один аргумент~--- парсер \lstinline|``p''|~--- и выполняет его мемоизацию. Ключевая часть мемоизации заключается в вызове функции \lstinline|memoresult| на
частично-примененном к входному потоку \lstinline|``p''|, об этой функции будет рассказано ниже. Для того, чтобы вызов \lstinline|memoresult| происходил только один раз в
реализации \lstinline|memo|, используется таблица для хранения результатов запусков. Реализация функций \lstinline|kmemo| и \lstinline|memo| не была изменена по сравнению с
библиотекой Meerkat, поэтому обратимся к функции \lstinline|memoresult|, которая претерпела некоторые изменения.

Алгоритм мемоизации в целом изменен не был~--- так же используются две таблицы: с продолжениями и с параметрами продолжений. При появлении нового продолжения происходит сохранение
его в соответствующую таблицу и вызов его на всех сохраненных параметрах. При вызове какого-либо продолжения частично-примененного парсера \lstinline|``p''| на новых параметрах происходит сохранение этих
параметров и вызов всех сохраненных продолжений на них:

\begin{lstlisting}[basicstyle=\small]
  let memoresult = fun p ->
    let ss : ('stream * 'a) list ref = ref      [] in
    let ks :                K.ks ref = ref K.empty in
    fun k ->
      if $\mbox{это первый запуск p}$
      then (
        ks := K.singleton k;
        p (fun a s ->
            $\mbox{если (s, a) не было в ss}$
            $\mbox{записываем ее в ss и}$
            $\mbox{запускаем на ней все продолжения из ks}$
            $\mbox{иначе возвращаем Empty}$
          )
      )
      else (
        ks := K.add k !ks;
        $\mbox{запустить k на всех сохраненных в ss парах}$
      )
\end{lstlisting}

Некоторые изменения, однако, были неизбежны. В качестве переменной \lstinline|``ss''| вместо списка входных потоков, на которых вызывались продолжения мемоизироваемого парсера, был использован список
пар из входного потока и монадического значения, потому что теперь продолжения вызываются на двух аргументах. Для получения структур, которые можно модифицировать в продолжениях были
использованы ссылки, как при описании таблицы \lstinline|``ss''|, так и \lstinline|``ks''|.

В реализации мемоизации был использован новый вид результата~--- \lstinline|Empty|. В данном случае некорректно возвращать \lstinline|Failure|, потому что никакой ошибки не произошло:
нет необходимости вызывать продолжение еще раз, и поэтому вызова не происходит. Именно об этом и говорит результат \lstinline|Empty|~--- о том, что никаких продолжений
вызвано не было, но это не связано с ошибкой, это связано с работой алгоритма.

Для хранения продолжений был разработан модуль \lstinline|K|, который помимо самого хранилища продолжений \lstinline|K.ks| предоставляет несколько полезных функций, например
функции \lstinline|K.singleton| и  \lstinline|K.add|, которые позволяют создавать новое хранилище с одним продолжением и добавлять продолжение в уже существующее хранилище
соответственно. Для правильного сравнения продолжений между собой необходимо использовать внутренний модуль \lstinline|Obj| и функцию неконтролируемого преобразования типов
\lstinline|Obj.magic|. Введение дополнительного модуля позволило инкапсулировать использование небезопасных примитивов.

\subsection{Реализация комбинатора \lstinline|opt|}

Помимо стандартных парсер-комбинаторов \lstinline|alt| и \lstinline|seq| в библиотеке Ostap используется еще несколько удобных комбинаторов. Одним из таких является комбинатор
\lstinline|opt|, который имеет один аргумент~--- парсер. На выходе \lstinline|opt| выдает парсер, который либо не делает ничего, либо совершает разбор с помощью парсера-аргумента,
и возвращает результат типа \lstinline|$\alpha$ option|.

В первую очередь хотелось бы обратить внимание на то, что этот комбинатор нельзя реализовать с помощью комбинатора \lstinline|alt| как показано ниже:

\begin{lstlisting}[basicstyle=\small]
  let opt p = alt p empty
\end{lstlisting}

Это связано с тем, что использование комбинатора \lstinline|alt| подразумевает, что типы обоих парсеров-аргументов и результирующего парсера должны совпадать,
но в случае \lstinline|opt| это не так. Правильная реализация такова:

\begin{lstlisting}[basicstyle=\small]
  let opt = fun p -> memo (fun s k ->
    let s' = Oo.copy s in
    let newk = kmemo (fun a s -> k (Some a) s) in
    (p s newk) <@> (k None s'))
\end{lstlisting}

По существу здесь использована специализированная версия реализации комбинатора \lstinline|alt|. Так как продолжение \lstinline|``k''| из-за неподходящего типа не может быть передано парсеру
\lstinline|``p''|, для его запуска используется новое продолжение \lstinline|``newk''|. В его реализации происходит необходимое изменение типа монадического значения
с некоторого $\alpha$ на \lstinline| $\alpha$  option| и мемоизация нового продолжения с помощью функции \lstinline|kmemo|. Затем это продолжение используется для
запуска парсера \lstinline|``p''|. Вариант пустого разбора комбинатора \lstinline|opt| реализуется с помощью вызова продолжения \lstinline|``k''| с монадическим значением \lstinline|None|,
которое показывает, что \lstinline|opt| не разобрал ничего.

\subsection{Явно-полиморфные типы}

Как уже было сказано в обзоре, входной поток в Ostap представляется с помощью объекта, и методы этого объекта вида \lstinline|getTOKEN| описывают лексический анализ
соответствующих лексем. Из-за того, что в типах методов объекта не может быть свободных переменных, возникла необходимость использовать явно-полиморфные типы:

\begin{lstlisting}[basicstyle=\small]
  < getCONST :
     $\beta$ . ($\alpha$ -> 'self -> ('self, $\beta$, $\gamma$) result) ->
               ('self, $\beta$, $\gamma$) result;
    .. >
\end{lstlisting}

В этом примере можно увидеть, что тип метода \lstinline|getCONST| явно-полиморфен по переменной $\beta$~--- типу пользовательской семантики. В отличии от переменной $\alpha$,
значение переменной $\beta$ невозможно установить лишь по этому методу, потому что ее значение будет известно только из продолжения, но его тип не установить заранее,
потому что пользовательская семантика не может ограничиваться принадлежностью к какому-либо типу. При реализации парсер-комбинаторов такой проблемы не возникало,
потому что тип пользовательской семантики просто был свободной переменной, однако в данном случае так сделать нельзя.

Помимо того, что необходимо явно прописывать типы у методов в описании входного потока, их также необходимо специфицировать при вызове.

\subsection{Генерируемая неподвижная точка}

Для того, чтобы таблицы мемоизации не создавались повторно в случае рекурсивных вызовов, необходимо использовать комбинатор неподвижной точки при описании парсеров.
Из-за того, что парсеры могут использовать друг друга в процессе разбора был использован генерируемый комбинатор неподвижной точки для нескольких функций. Продемонстрируем
это на следующем примере:

\begin{lstlisting}[basicstyle=\small]
    ostap (x : "a"; y[z] : y[z] "+" x | x)

                $\Downarrow$

    let (x, y) =
      let generated_fixpoint = ... in
      let x2   = fun x y -> ostap ("a")
      and y2 z = fun x y -> ostap (y[z] "+" x | x)
    in generated_fixpoint x2 y2
\end{lstlisting}

Здесь описывается два парсера \lstinline|``x''| и \lstinline|``y''|, и поэтому генерируется двуместный комбинатор неподвижной точки. Помимо этого для работы комбинатора неподвижной
точки необходима генерация освобожденных от рекурсивных вызовов парсеров \lstinline|``x2''| и \lstinline|``y2''|. Как можно видеть, функция \lstinline|``x2''| имеет два дополнительных
аргумента по сравнению с \lstinline|``x''|, которые для простоты называются \lstinline|``x''| и \lstinline|``y''|. При этом тела функций \lstinline|``x''| и \lstinline|``x2''| никак не отличаются.
Таким образов все рекурсивные вызовы парсеров \lstinline|``x''| и \lstinline|``y''| в описании этих самых парсеров заменены на вызовы аргументов \lstinline|``x2''| и  \lstinline|``y2''|.
Именно в таком виде комбинатор неподвижной точки принимает на вход обрабатываемые функции.

С помощью генерируемого комбинатора неподвижной точки получилось избавиться и от еще одной проблемы: в случае использования параметризованных парсеров и при повторном вызове на
одном и том же параметре таблицы мемоизации создавались повторно. Для решения этой проблемы внутри комбинатора неподвижной точки генерируется код для мемоизации парсера по всем
его параметрам. Этот код проверяет, вызывался ли парсер высшего порядка на текущем параметре, и берет уже полученное и сохраненное значение в случае повторного вызова.
Таким образом, комбинатор неподвижной точки генерируется не только в зависимости от количества описываемых парсеров, но и в зависимости от количества параметров у каждого из них.

\section{Тестирование и апробация}

Описанная в предыдущих разделах модификация монадических парсер-комбинаторов обладает следующими преимуществами по сравнению с исходной реализацией:

\begin{itemize}
  \item поддержкой произвольного порядка альтернатив без использования правила ``longest match first'';
  \item поддержкой левой рекурсии;
  \item более высокой производительностью за счет мемоизации.
\end{itemize}

Все эти свойства были подтверждены экспериментально.

В первую очередь было проведено регрессионное тестирование на стандартном наборе тестов, входящих в состав оригинальной библиотеки Ostap. Это тестирование показало,
что новая реализация парсер-комбинаторов полностью покрывает функциональность исходной библиотеки. При этом тесты, написанные с применением синтаксического расширения (то есть
используя более высокоуровневый интерфейс, чем примитивные комбинаторы) вообще не подвергались никакой модификации. Тесты же на индивидуальные комбинаторы были модифицированы
незначительно~--- к вызовам парсеров на самом верхнем уровне были добавлены тривиальные продолжения.

Помимо регрессионных тестов для проверки функциональности был использован синтаксический анализатор учебного языка, используемого в курсе
компиляторов\footnote{https://compscicenter.ru/courses/compilers/2018-spring}. Эталонный компилятор, собранный с помощью новой реализации библиотеки, прошел все
регрессионные тесты без модификации. Таким образом было подтверждено отсутствие регресса по сравнению с исходной библиотекой.

\subsection{Нечувствительность с правилу ``longest match first''}

В эталонном компиляторе, упомянутом выше, есть фрагмент, где нарушение правила ``longest match first'' приводит к возникновению труднообнаружимых отложенных ошибок.
Этот фрагмент локализован в реализации парсера выражений:

\begin{lstlisting}[basicstyle=\small]
  Ostap.Util.expr [|
    `Lefta, [$\verb/"!!"/$];
    `Lefta, [$\verb/"&&"/$];
    `Nona , [$\verb/"=="/$; $\verb/"!="/$; $\verb/"<="/$; $\verb/"<"/$; $\verb/">="/$; $\verb/">"/$];
    `Lefta, [$\verb/"++"/$; $\verb/"+"/$; $\verb/"-"/$];
    `Lefta, [$\verb/"*"/$; $\verb/"//"$; $\verb/"%"/$] |]
\end{lstlisting}

В данном фрагменте используется утилита \lstinline|Util.expr|. Она позволяет описывать выражения
с бинарными и унарными операциями разного старшинства и ассоциативности с помощью простых конструкций. Строка, начинающаяся с \lstinline|`Nona|, описывает неассоциативные
операции в выражениях. В реализации утилиты \lstinline|Util.expr| она будет преобразована в несколько последовательных альтернатив из парсеров соответствующих
строк-элементов массива. Поэтому в данном случае раньше необходимо было следовать правилу ``longest match first'', хотя использование альтернативы и скрыто от пользователя
с помощью утилиты \lstinline|Util.expr|. На практике это означало, что порядок знаков операций имел значение: например, нельзя было указать знак операции ``<'' \emph{раньше}
знака операции ``<='', поскольку строка ``<'' является подстрокой ``<=''. То же самое было верно и для знаков ``++'' и ``+'' ниже по тексту фрагмента.

Для проверки того, что от правила ``longest match first'' можно отказаться при использовании реализации парсеров в новой версии библиотеки, знаки ``>'' и ``>='' были
переставлены местами. Оказалось, что, действительно, такие парсеры работают, причем работают с одинаковой скоростью для любого порядка знаков.

\subsection{Использование леворекурсивных описаний}

Описание оператора ``;'' также явняется интересным местом в реализации парсера эталонного компилятора:

\begin{lstlisting}[basicstyle=\small]
  parse:
        s:stmt ";" ss:parse {Seq (s, ss)}
      | stmt;
\end{lstlisting}

В этом листинге представлена реализациия парсера \lstinline|parse|, который анализирует последовательности вида \lstinline|``stmt; stmt; ...; stmt''|. При использовании старой
версии Ostap необходимо было использовать именно эту нелеворекурсивную реализацию~--- самым левым нетерминалом должен был быть \lstinline|stmt|, а не \lstinline|parse|. Реализация,
представленная ниже, ранее не работала именно по этой причине:

\begin{lstlisting}[basicstyle=\small]
  parse:
        ss:parse ";" s:stmt {Seq (ss, s)}
      | stmt;
\end{lstlisting}

Здесь парсер \lstinline|parse| описан леворекурсивно~--- самым левым нетерминалом является \lstinline|parse|~--- поэтому происходило зацикливание алгоритма разбора.

Для проверки новой версии библиотеки была использована вторая реализация, при этом парсер остался работоспособным и время его работы не изменилось.

\subsection{Повышение производительности}

Для сравнения производительности старой и новой версии библиотеки была использована та же реализация эталонного компилятора, что и в предыдущих разделах. В качестве тестового
набора использовались сгенерированные программы вида \lstinline|``x1 := {expr}''| для разной длины выражения.  Были проведены также замеры и для строк с несколькими
выражениями, то есть вида \lstinline|``x1 := {expr}; ... xn := {expr}''|, однако оказалось, что время работы на таких входах пропорционально времени работы для одного присваивания,
поэтому для оценки производительности такие тесты далее не использовались. Тем не менее тестирование на нескольких присваиваниях выявило интересное свойство новой реализации:
увеличение числа присваиваний пропорционально увеличивает глубину рекурсивных вызовов для новой версии (при использовании старой версии такого не наблюдается). Это связано с тем,
что новая версия использует гораздо более глубокие рекурсивные вызовы, так как парсер-комбинатор \lstinline|seq| реализован таким образом, что второй парсер запускается в продолжении,
а значит требует дополнительное место на стеке. Поэтому при использовании новой версии необходимо правильно конфигурировать среду времени исполнения для выделения достаточного
количества памяти под стек.

Результаты апробации показаны в таблице ниже. Видно, что парсеры, написанные с помощью новой версии библиотеки, работают в несколько раз быстрее по сравнению со старой версией.
В первом столбце показан размер выражения (число знаков бинарных операций), а в двух других представлено время работы в секундах парсера эталонного компилятора, собранного с помощью старой версии библиотеки, и того же парсера, собранного с помощью новой версии.

\begin{table}[htbp]
\begin{center}
\begin{tabular}{|c|c|c|}
\hline
\textbf{Размер} & \multicolumn{2}{|c|}{\textbf{Время разбора, cек.}} \\
\cline{2-3}
\textbf{примера} & \textbf{Старая версия} & \textbf{Новая версия} \\
\hline
100& 0,25& 0,13 \\
\hline
200& 0,52& 0,24 \\
\hline
400& 2,13& 0,56 \\
\hline
800& 9,19& 1,97 \\
\hline
1200& 21,93& 4,61 \\
\hline
1600& 42,65& 9,50 \\
\hline
\end{tabular}
\vskip3mm
%\caption{Результаты сравнения производительности старой и новой реализаций библиотеки}
\label{tab1}
\end{center}\vspace{-10mm}
\end{table}


\section{Заключение}

В рамках данной работы была выполнена реализация монадических парсер-комбинаторов, использующих пользовательскую семантику. На основе алгоритма авторов библиотеки Meerkat был разработан адаптированный алгоритм, который далее был интегрирован в библиотеку Ostap.

\begin{thebibliography}{00}
\bibitem{meerkat} A. Izmaylova, A. Afroozeh, and T. van der Storm T. Practical, General Parser Combinators // PEPM '16 Proceedings of the 2016 ACM SIGPLAN Workshop on Partial Evaluation and Program Manipulation, pp. 1--12, 2016.
\bibitem{ostap} D. Boulytchev. Ostap: Parser Combinator Library and Syntax Extension for Objective Caml, 2009.
\bibitem{boulytchev2015combinators} D. Boulytchev. Combinators and Type-Driven Transformers in objective caml // Science of Computer Programming, Vol. 114, Elsevier, pp. 57--73, 2015.
\bibitem{meijer} G. Hutton and E. Meijer. Monadic Parser Combinators // Technical Report NOTTCS-TR-96-4, Department of Computer Science, University of Nottingham, 1996.
\bibitem{wadler} Ph. Wadler. Monads for functional programming // Proc. Marktoberdorf Summer school on program design calculi, 1992.
\bibitem{tratt} L. Tratt. Direct Left-Recursive Parsing Expression Grammars // Technical Report EIS-10-01, Middlesex University, 2010.
\bibitem{frost} R. A. Frost, R. Hafiz, and P. Callaghan. Parser Combinators for Ambiguous Left-Recursive Grammars // Practical Aspects of Declarative Languages, PADL’08, 2008.
\bibitem{warth} A. Warth, J. R. Douglass, and T. Millstein. Packrat Parsers Can Support Left Recursion // Partial Evaluation and Semantics-based Program Manipulation, PEPM ’08, pp. 103--110, 2016.
\bibitem{ford2002packrat} B. Ford. Packrat parsing: simple, powerful, lazy, linear time, functional pearl // ACM SIGPLAN Notices, Vol. 37, No. 9, ACM, 2002.
\bibitem{ford2004parsing} B. Ford. Parsing expression grammars: a recognition-based syntactic foundation // ACM SIGPLAN Notices, Vol. 39, No. 1, ACM, 2004.
\bibitem{swierstra2001combinator} S. D. Swierstra. Combinator parsers: From toys to tools // Electronic Notes in Theoretical Computer Science, Vol. 41, No. 1, Elsevier, pp. 38--59, 2001.
%\bibitem{knuth1968semantics} D. E. Knuth. Semantics of context-free languages // Mathematical systems theory, Vol. 2, No. 2, Springer, pp. 127--145, 1968.
\bibitem{earleycomb} D. Peake, Sean Seefried. A Combinator Parser for Earley's Algorithm, 2004.
\bibitem{GLLcomb} D. Spiewak. Generalized Parser Combinators, 2010.
\bibitem{memo} M. Johnson. Memoization in Top-down Parsing // Comput. Linguist., Vol.~21, No~3, 1995.
\bibitem{leftcorner} A.Baars, D.Swierstra, M.Viera, Marcos. Typed Transformations of Typed Grammars: The Left Corner Transform // Electronic Notes in Theoretical Computer
Science,  Vol.~253, No~7, 2010.
\end{thebibliography}

% \section{Ответ на замечания}
% В данной дополнительной главе будут описаны вопросы, которые появились у рецензентов и слушателей на конференции, а также мои ответы на них. Выражаю свою благодарность слушателям, авторам вопросов в частности, и рецензентам за полезные замечания и отзывы.
% \subsection{Замечания рецензентов}
% Все замечания, касающиеся стилистики текста, использования неудачных выражений и т.п., были учтены и повлекли необходимые поправки в тексте статьи. Одно из замечаний касалось апробации и ставило под вопрос сам процесс проведенной апробации, так как для этого необходимо не только разработать парсер-комбинаторы, но и использовать их в инструментально среде. В данном случае такой инструментальной средой выступала библиотека Ostap, с помощью которой был разработан синтаксический анализатор учебного языка, и замеры производительности проводились уже этого анализатора. Поэтому в процессе апробации, описанной в статье, с этой точки зрения все корректно. Еще одним недостатком данной работы, который был отражен в отзывах рецензентов, является отсутствие ассимптотической оценки предложенного решения. Это связано с тем, что такая оценка уже была проведена в статье~\cite{meerkat}. Самая основная часть алгоритма осталась неизменной при адаптации к монадическим парсерам, использующих пользовательскую семантику, и парсерам высшего порядка. Поэтому было решено не описывать эту оценку в статье для того, чтобы уделить время другим более важным и интересным аспектам данной работы, а не описывать то, что уже было описано в других статьях. Еще одно замечание касалось того, что я не оценил по отдельности вклад мемоизации и моих изменений в общее увеличение производительности. Как мне кажется, это невозможно сделать, так как без этих изменений мемоизация просто не будет работать для парсеров высших порядков, поэтому невозможно получить какие-либо оценки производительности мемоизации в чистом виде для монадических парсеров с пользовательской семантикой. Мои изменения в реализацию парсер-комбинаторов были совершены как на уровне типов, так и на самом высоком уровне библиотеки -- на уровне синтаксического расширения, и все они были необходимы для корректной работы разработанных парсер-комбинаторов.
% \subsection{Вопросы с конференции}
% Первый из вопросов, заданных мне на конференции, касался, насколько я понял автора вопроса, использования комбинатора неподвижной точки и сложности интегрирования этого алгоритма комбинаторов в другие системы. При использовании новой версии библиотеки Ostap нет необходимости задумываться о неподвижной точки: этот программный код автоматически генерируется и совершенно скрыт от пользователя. В этом заключалось и заключается одно из достоинств библиотеки Ostap: трудности использования парсер-комбинаторов скрыты от пользователя под простыми конструкциями, похожими на EBNF. При интеграции в другие системы, также нуобходимо каким-то образом и реализовывать неподвижную точку с мемоизацией по параметрам для поддержки парсеров высших порядков. Еще один вопрос, который мне задали, касался вопроса затрачиваемой памяти при работе анализаторов, сконструированных при помощи новой версии библиотеки Ostap. Эти данные не были представлены в статье и, как следствие, в презентации по той причине, что ничего интересного в них не наблюдается, и, кроме самих данных, и показать нечего. Да, затраты возросли, но по времени новые парсер-комбинаторы дают больший выигрыш. При этом количество таблиц линейно зависит как от размера грамматики, так и он длины входного потока, поэтому потребление памяти не может выйти на первый план при использовании новой версии библиотеки Ostap.

\end{document}
